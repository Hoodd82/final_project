% Options for packages loaded elsewhere
\PassOptionsToPackage{unicode}{hyperref}
\PassOptionsToPackage{hyphens}{url}
%
\documentclass[
]{article}
\usepackage{amsmath,amssymb}
\usepackage{lmodern}
\usepackage{ifxetex,ifluatex}
\ifnum 0\ifxetex 1\fi\ifluatex 1\fi=0 % if pdftex
  \usepackage[T1]{fontenc}
  \usepackage[utf8]{inputenc}
  \usepackage{textcomp} % provide euro and other symbols
\else % if luatex or xetex
  \usepackage{unicode-math}
  \defaultfontfeatures{Scale=MatchLowercase}
  \defaultfontfeatures[\rmfamily]{Ligatures=TeX,Scale=1}
\fi
% Use upquote if available, for straight quotes in verbatim environments
\IfFileExists{upquote.sty}{\usepackage{upquote}}{}
\IfFileExists{microtype.sty}{% use microtype if available
  \usepackage[]{microtype}
  \UseMicrotypeSet[protrusion]{basicmath} % disable protrusion for tt fonts
}{}
\makeatletter
\@ifundefined{KOMAClassName}{% if non-KOMA class
  \IfFileExists{parskip.sty}{%
    \usepackage{parskip}
  }{% else
    \setlength{\parindent}{0pt}
    \setlength{\parskip}{6pt plus 2pt minus 1pt}}
}{% if KOMA class
  \KOMAoptions{parskip=half}}
\makeatother
\usepackage{xcolor}
\IfFileExists{xurl.sty}{\usepackage{xurl}}{} % add URL line breaks if available
\IfFileExists{bookmark.sty}{\usepackage{bookmark}}{\usepackage{hyperref}}
\hypersetup{
  pdftitle={Final project documentation - PDA},
  hidelinks,
  pdfcreator={LaTeX via pandoc}}
\urlstyle{same} % disable monospaced font for URLs
\usepackage[margin=1in]{geometry}
\usepackage{graphicx}
\makeatletter
\def\maxwidth{\ifdim\Gin@nat@width>\linewidth\linewidth\else\Gin@nat@width\fi}
\def\maxheight{\ifdim\Gin@nat@height>\textheight\textheight\else\Gin@nat@height\fi}
\makeatother
% Scale images if necessary, so that they will not overflow the page
% margins by default, and it is still possible to overwrite the defaults
% using explicit options in \includegraphics[width, height, ...]{}
\setkeys{Gin}{width=\maxwidth,height=\maxheight,keepaspectratio}
% Set default figure placement to htbp
\makeatletter
\def\fps@figure{htbp}
\makeatother
\setlength{\emergencystretch}{3em} % prevent overfull lines
\providecommand{\tightlist}{%
  \setlength{\itemsep}{0pt}\setlength{\parskip}{0pt}}
\setcounter{secnumdepth}{-\maxdimen} % remove section numbering
\ifluatex
  \usepackage{selnolig}  % disable illegal ligatures
\fi

\title{Final project documentation - PDA}
\author{}
\date{\vspace{-2.5em}}

\begin{document}
\maketitle

{
\setcounter{tocdepth}{2}
\tableofcontents
}
\hypertarget{general}{%
\section{General}\label{general}}

\hypertarget{data-requirements}{%
\subsubsection{Data Requirements}\label{data-requirements}}

The data you use in the project has to meet the following criteria:

\begin{itemize}
\tightlist
\item
  At least 5,000 rows
\item
  At least 3 sources of data
\item
  Must contain text data, numeric data and dates
\end{itemize}

Most of the data supplied with the briefs do meet the requirements, but
we recommend that you personally make sure that you are indeed meeting
these criteria. If your current datasets do not, you are more than
welcome to supplement your analysis with further datasets that would
help you meet the requirements.

\hypertarget{product-requirements}{%
\subsubsection{Product Requirements}\label{product-requirements}}

Your actual final project analysis process will remain largely the same
regardless if you are completing the PDA or not. However, you will need
to do a bit more formal planning/documentation of your project for the
PDA.

Please create a project plan/documentation according to the template
below:

\hypertarget{template}{%
\section{Template}\label{template}}

Use this template with associated subheadings to create the
documentation to make sure you cover all necessary learning outcomes for
the PDA. If you have any questions, just let an instructor know.

\hypertarget{context}{%
\subsection{Context}\label{context}}

\hypertarget{business-intelligence-and-data-driven-decision-making}{%
\paragraph{Business intelligence and data-driven decision
making}\label{business-intelligence-and-data-driven-decision-making}}

What insights can the business/organisation gain from your analysis and
how will your analysis help the business/organisation make better
decisions?

\hypertarget{domain-knowledge-and-the-business-context}{%
\paragraph{Domain knowledge and the business
context}\label{domain-knowledge-and-the-business-context}}

I was hired by Newark International Airport to investigate the effect of
weather on flight departure delays.

The business believes that poor weather conditions are causing too many
delays and want to invest in improving facilities, so that aircraft can
take off in more types of weather. However, they do not fully understand
how serious weather related delays are, and are not sure what type of
weather they should be most concerned about. As part of investigating
the effect of weather, other factors were explored to understand how
important weather is in comparison to them.

Newark Airport also want to understand how they compare to other New
York airports.

\hypertarget{data}{%
\subsection{Data}\label{data}}

\hypertarget{internal-and-external-data-sources}{%
\paragraph{Internal and external data
sources}\label{internal-and-external-data-sources}}

The data received was entirely sourced from the business, no external
data sources were used in the project.

\hypertarget{types-of-data}{%
\paragraph{Types of data}\label{types-of-data}}

planes.csv: tailnum - character year - integer type - character
manufacturer - character model - character engines - integer seats -
integer speed - integer engine - character

airports.csv: faa - character name - character lat - numeric lon -
numeric alt - numeric tz - numeric dst - character tzone - character

airlines.csv: carrier - character name - character

flights.csv: year - integer month - integer day - integer dep\_time -
integer sched\_dep\_time - integer dep\_delay - numeric arr\_time -
integer sched\_arr\_time - integer arr\_delay - numeric carrier -
character flight - integer tailnum - character origin - character dest -
character air\_time - numeric distance - numeric hour - numeric minute -
numeric time\_hour - POSIXct / POSIXt

weather: origin - character year - integer month - integer day - integer
hour - integer temp - numeric dewp - numeric humid - numeric wind\_dir -
numeric wind\_speed - numeric wind\_gust - numeric pressure - numeric
visib - numeric time\_hour - POSIXct / POSIXt

\hypertarget{data-formats}{%
\paragraph{Data formats}\label{data-formats}}

All data used in the project was in the form of .csv files.

\begin{itemize}
\tightlist
\item
  aircraft.csv
\item
  airlines.csv
\item
  flights.csv
\item
  planes.csv
\item
  weather.csv
\end{itemize}

\hypertarget{data-quality-and-bias}{%
\paragraph{Data quality and bias}\label{data-quality-and-bias}}

The data provided related only to flights flying internally within the
United States. It would not therefore be possible to draw any
conclusions from the project findings regarding international flights.

\hypertarget{ethics}{%
\subsection{Ethics}\label{ethics}}

\hypertarget{ethical-issues-in-data-sourcing-and-extraction}{%
\paragraph{Ethical issues in data sourcing and
extraction}\label{ethical-issues-in-data-sourcing-and-extraction}}

Technically the data used for this project does not fall within the
scope of GDPR as it relates to flights taken internally within the
United States, who do not have a federal privacy laws like GDPR.
Additionally, there is no information relating to passengers, only the
movement of flights.

\hypertarget{ethical-implications-of-business-requirements}{%
\paragraph{Ethical implications of business
requirements}\label{ethical-implications-of-business-requirements}}

There are no ethical implications of the business requirements, which is
to improve the facilities at Newark International Airport.

\hypertarget{analysis}{%
\subsection{Analysis}\label{analysis}}

\hypertarget{stages-in-the-data-analysis-process}{%
\paragraph{Stages in the data analysis
process}\label{stages-in-the-data-analysis-process}}

\begin{itemize}
\tightlist
\item
  Understand the scope of the project and the problem.
\item
  Data cleaning
\item
  Exploratory data analysis
\item
  Model building
\item
  Reporting
\end{itemize}

\hypertarget{tools-for-data-analysis}{%
\paragraph{Tools for data analysis}\label{tools-for-data-analysis}}

This project was carried out entirely using R.

\hypertarget{descriptive-diagnostic-predictive-and-prescriptive-analysis}{%
\paragraph{Descriptive, diagnostic, predictive and prescriptive
analysis}\label{descriptive-diagnostic-predictive-and-prescriptive-analysis}}

\begin{itemize}
\tightlist
\item
  Linear regression: Linear regression is predictive analysis method
  which is a way of predicting future events between a dependent
  variable, in this case departure delays, and independent or predictor
  variables.
\item
  Decision trees \& random forests: Decisions trees and random forests
  are both predictive modeling methods.
\end{itemize}

\end{document}
